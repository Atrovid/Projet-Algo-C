\documentclass[10pt,a4paper]{article}
\usepackage[utf8]{inputenc}
\usepackage[french]{babel}
\usepackage[T1]{fontenc}
\usepackage{amsmath}
\usepackage{amsfonts}
\usepackage{amssymb}
\usepackage{graphicx}
\usepackage[lmargin=2cm,rmargin=2cm,tmargin=2cm,bmargin=2.5cm]{geometry}

\usepackage{ae}	
\usepackage{gensymb}

\usepackage{hyperref}
\usepackage{tabularx}



\usepackage{xcolor}
\usepackage{fancyhdr}
\pagestyle{fancy}

\renewcommand{\headrulewidth}{1pt}
\renewcommand{\footrulewidth}{1pt}
\fancyhead[L]{Cahier de projet}
\fancyhead[R]{Projet Algo \& C}
\fancyfoot[L]{Professeurs encadrants : \\Luc Brun et Eric Ziad}
\fancyfoot[R]{ENSICAEN - 1A Informatique\\ Année scolaire : 2021-2022}

\usepackage{lmodern}
\usepackage{ae}
\usepackage{enumerate}

\def\sqw{\hbox{\rlap{\leavevmode\raise.3ex\hbox{$\sqcap$}}$%
\sqcup$}}
\def\sqb{\hbox{\hskip5pt\vrule width4pt height6pt depth1.5pt%
\hskip1pt}}
\newcommand{\hsp}{\hspace{20pt}}
\newcommand{\HRule}{\rule{\linewidth}{0.5mm}}




\begin{document}

\tableofcontents
\newpage
\part{Semaine 1}
\section{Séance 25/11/2021 (début 13h)}
\subsection{Quelques informations}
\begin{itemize}

\item Méthode d'inversion de table : admettons qu'on a une image avec un pixel qui est vert. Admettons aussi que notre table de couleur est composé de rouge et bleu. \\
Le pixel va se transformer en la couleur la plus proche entre le rouge et le bleu. \\

\item Le dossier ressource est à cherche dans la racine sur le serveur ssh (voir moodle pour plus d'information)\\
\item Installer GIMP (pour ouvrir les images)
\end{itemize}
\subsection{Arborescence du dossier (dans le gros dossier Projet Algo \& C}
\begin{itemize}
\item Un sous fichier src
\item Un sous fichier test (pour minunit)
\item Un sous fichier ressource (qui ont été mis à disposition sur Cybele)
\item Un sous fichier doc (pour la génération de la documentation avec Doxygen)
\item Un (sous fichier??) pour le makefile
\item Un sous dossier include pour unclure tout les fichiers .h (headers)
\end{itemize}

\subsection{Fichier image.c}

\subsubsection{Comprendre le dossier fichier image.c}
\subsubsection{Liste des fonctions et ce qu'ils font}

\begin{itemize}
\item void DEFAIRE\_image (image self) --> Initialise une image vide
\item 
\end{itemize}
\subsubsection{Liste des structures}
\subsection{Implementation de la méthode triviale}
\begin{itemize}
\item Pour charger une image, il faut d'abord initialiser une image vide, pour cela, il faut qu'on utilise la fonction image\_initialize
\end{itemize}
\end{document}